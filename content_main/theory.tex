\chapter{Theoretical Background}

\section{Introduction}
\nomenclature[A]{$c$}{Speed of light in a vacuum inertial system \nomunit{$299,792,458\, m/s$}}
The analog signal $s(t)$ being send out of the pulse generator is defined as the sum of all pulses send out, and defined as
\begin{equation}\label{eq:analog_pulse_signal}
s(t) = \sum_{i=0}^{\infty} p(t-iT)cos(\omega_c(t-iT)),\quad i = 0,1,2,...,\infty
\end{equation}

\nomenclature[R]{$s(t)$}{Analog signal sent from pulse generator}
\nomenclature[R]{$t$}{Global Time \nomunit{s}}
\nomenclature[R]{$\lambda$}{Wavelength \nomunit{m}}

\nomenclature[R]{$p(t)$}{Gaussian pulse}
\nomenclature[R]{$\omega_c$}{Angular carrier frequency}
\nomenclature[R]{$f_c$}{Carrier frequency \nomunit{Hz}}

\nomenclature[R]{$f_p$}{\acrlong{prf} (\acrshort{prf}) \nomunit{Hz}}

\nomenclature[R]{$T$}{Pulse Repetition Period \nomunit{s}}

\nomenclature[R]{$\tau$}{Time delay \nomunit{s}}


Where $\omega_c$ is the angular carrier frequency, defines as $\omega_c= 2 \pi f_c$ and $f_c$ is the carrier frequency. The function $p$ represents the Gaussian pulse which is defined as
\begin{equation}
    p(t) = \exp\left( -\frac{(t-\frac{T_p}{2})^2}{(2\sigma)^2} \right)
\end{equation}
Typical values of $\sigma$ and pulse duration $T_p$ are $0.2$ ns and $1$ ns respectively.

\begin{equation}
    s_i = (t_d) = p(t_d)cos(\omega_c t_d)
\end{equation}
\nomenclature[R]{$s_i(t_d)$}{$i$th analog pulse sent}
\nomenclature[R]{$t_d$}{Time variable within the time interval (0,T) \nomunit{s}}

Signal received by the antenna, ignoring attenuation, distortion and noise.

\begin{equation}\label{eq:receiver_signal}
    r_i(t_d) = s_i(t_d-\tau) = p(t_d-\tau)cos(\omega_c(t_d-\tau)) 
\end{equation}
\nomenclature[R]{$r_i(t_d)$}{$i$th analog pulse received by the antenna}

where $\tau$ is given as
\begin{equation}
\tau = \frac{2R(t)}{c} = \frac{2R(t_d+iT)}{c}    
\end{equation}


Substituting equation \ref{eq:analog_pulse_signal} into equation \ref{eq:receiver_signal}.


\section{Doppler Effect in Radar}

\begin{equation}
    f_r = \left(\frac{1+\frac{v_r}{c}}{1-\frac{v_r}{c}}\right) f_0
\end{equation}
where $v_r$ is the radial velocity of the scatter with respect to the source.
\begin{equation}
    f_r = \left[ 1+2\frac{v_r}{c} \right]
\end{equation}

Difference between the transmitted frequency $f_c$ and the received frequency $f_r$ is the Doppler shift $f_D$,
\begin{equation}
f_r-f_c = f_D =\frac{2v_r}{c}f_c = \frac{2 v_r}{\lambda}
\end{equation}
$v_r$ is positive for approaching target and negative for receding target.
\nomenclature[R]{$v_r$}{Radial velocity of target}


\begin{equation}
    f_D = \frac{2v}{\lambda_0}cos \psi
\end{equation}

\section{Extraction of Doppler Information}



\subsection{Video Detectors and Coherent Detectors}

\begin{equation}
s(t) = A cos(2\pi \omega )    
\end{equation}


\subsubsection{Mixers}
\begin{itemize}
    \item Combine two frequencies (mixing) and create a new frequency.
    \item Uses a non-linear signal processing solution
    \item Hetrodyning: shift one frequency range into another, new one
    
\end{itemize}

\subsubsection{Complex Base-band Signal}



\section{The Micro-Doppler Effect}


\section{Tracking}

Tracking consists of the following steps:
\begin{enumerate}
    \item plot extraction
    \item track initiation
    \item plot-track association
    \item track deletion
\end{enumerate}